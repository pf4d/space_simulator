\input{functions.tex}
\DeclareMathOperator*{\argmax}{arg\,max}

\begin{document}
\small

\title{Space Simulator - Project 05}
\author{Evan Cummings and Joshua Bartz\\
CSCI 576 - Human Computer Interaction}

\maketitle

\section{Test script}

Hello, and welcome to Space Simulator, a new and innovative space-flight simulator.  We'd like to take this opportunity to thank you for volunteering your time to help us test this new and exciting adventure.  I will be reading from this script to ensure consistency between all of our participants.

Now a little about Space Simulator.  Space Simulator is a simulation of what it would be like to actually pilot a craft in outer space, using realistic physics.  Please note that Space Simulator is still in its development stage and that it may break down under certain circumstances.  If that does happen, we will restart the program. 
  
Understand that this exercise is to test the product and its usability and in no way implies your abilities.  If at any time you feel uncomfortable, please inform us and we will terminate the exercise immediately. (We will be recording this exercise so that we can gather as much information as possible from this session).
   
Your opinion and ideas are important to us.  Whenever possible, please speak your thoughts freely.  Do not be concerned about offending us.  If you forget to think aloud, I'll remind you to keep talking. 
    
As you’re working through Space Simulator, I won’t be able to provide help or answer questions.  This is because we want to create the most realistic situation possible.  Even though I won't be able to answer your questions during the exercise, please ask them. We'll note your questions and answer them at the end of the exercise. 
     
Do you have any questions? 

\section{Task list}

\begin{enumerate}

  \item \textbf{Control familiarization!}  A comprehensive walk-through of all of the directional controls used to pilot the ship. 

  Tasks:
  \begin{itemize}
    \item Launch the game by typing ``\texttt{python spaceSimulator.py}''
    \item Pause the game
    \item Review controls
    \item Resume game
    \item Accelerate the ship forward until the velocity vector reads at least 15.0
    \item Bring the ship to a stop
    \item Accelerate in the left direction until the velocity vector reads at least 15.0
    \item Bring the ship to a stop
    \item Repeat for ascend, roll left, and yaw left
    \item Exit the game
  \end{itemize}
  
  \item \textbf{Collision correction.}  You will begin a new game, then attempt to fly into the ``planet.''  After the collision, your ship will behave erratically, and you will need to fix it.
  
  Tasks:
  \begin{itemize}
    \item Launch the game by typing ``\texttt{python spaceSimulator.py}''
    \item Locate the planet by any means
    \item Pilot the ship into a collision course with the planet
    \item Collide with the planet
    \item Take note of the force vectors acting on the craft
    \item One direction at a time, bring the ship to a complete stop
    \item Exit the game
  \end{itemize}
  
  \item \textbf{Orbital slingshot.}  Using the gravity of the planet and the moon, fly around the moon and let it ``slingshot'' you away.
  
  Tasks:
  \begin{itemize}
    \item Launch the game by typing ``\texttt{python spaceSimulator.py}''
    \item Locate the star and planet by any means
    \item Pilot the ship near the planet, aiming slightly away from it
    \item As you prepare to pass, ensure the planet is moving in a direction that is as close to perfectly opposite as your own as possible
    \item As you enter orbit of the planet, ensure you maintain a great enough speed to prevent being pulled into the surface
    \item Allow the planet's gravity to swing you around it
    \item As the direction of the ship becomes closer to that of the planet, note the increase in speed until the ship breaks orbit
    \item Exit the game
  \end{itemize}

  \item \textbf{Experiment!}  Time for some free-flying.  Taking the controls you have learned, fly for 3-5 minutes. 
  
  Tasks:
  \begin{itemize}
    \item Launch the game by typing ``\texttt{python spaceSimulator.py}''
    \item For 3-5 minutes, practice using the controls
    \item Attempt near-misses with the planet and star
    \item See how fast you can move the ship in multiple directions without losing control
    \item See how long you can maintain orbit of the planet
  \end{itemize}

\end{enumerate}

\end{document}


