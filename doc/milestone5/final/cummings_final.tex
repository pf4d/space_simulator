\input{functions.tex}
\usepackage[top=1in, bottom=1in, left=1in, right=1in]{geometry}

\begin{document}

\title{Space Simulator - user testing reflection}
\author{Evan Cummings\\
CSCI 576 - Human Computer Interaction}

\maketitle

The user-testing process was a useful experience which highlighted both positive and negative aspects of our design.  The first two tests included dual participants and were conducted in my apartment; these were a bit more informal than the last two conducted in my office,  and as a result these participants displayed increased relaxation, immersion in the gameplay, entertainment, and communication.  Therefore, any future tests shall include pairs of players in an informal environment.

Two of the three singular participants displayed signs of boredom and disinterest after becoming frustrated with the game, with one even asking to stop the test.  I feel that at this stage of development, the user needs to have a genuine interest in the subject matter and a strong will to succeed.  Players who are easily discouraged or shy may be overwhelmed by the plethora of controls and become emotionally incapable of proceeding.  I believe that pairs of users will help mitigate this problem.  For example, a weaker player might be challenged by a stronger player to correct a navigation issue and as such become more motivated to continue trying until they succeed.

There were many issues with the control scheme and navigation meters.  Before any more testing is performed I will include a system providing participants with an option to customize their controls and receive help on the heads-up-display.  A more complete tutorial mode would also be helpful.  One can envision an entire game designed around tutorials which guide the player into space-simulator tasks such as docking with a space station, attaining orbit with a planet, or flying to the moon.

In order to decrease the difficulty of the game, we decreased the game speed for the final two participants by about 50\%.  This made the game much more easy to play, and is in my opinion the main cause of the complete success of participant number four.  With very little training and need to look at the controls, he was able to complete all the tasks -- and complete them faster -- than all the other participants.  When it came to the final task, free-flight, he attempted attaining orbit without being asked, and very easily maneuvered the craft to a close proximity of the planet.

In conclusion, I believe that the tests went overall quite well.  I was expecting much more difficulty for the users to successfully pilot the craft, and was pleasantly surprised to see the majority of players succeeding in their tasks.  While difficult to keep the player's attention, we have successfully identified problems, collected feedback, and increased the game's playability.


\end{document}


